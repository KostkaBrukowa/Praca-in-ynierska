% Źródło innego typu, np. repozytorium na GitHubie.
@manual{ref_heuristic_alg,
    title        = "Rekoncyliacja w bibliotece React",
    note  = "Dostęp zdalny (16.01.2021): \url{https://pl.reactjs.org/docs/reconciliation.html}"
}

@manual{ref_apollo_doc,
    title = "Apollo Client",
    note  = "Dostęp zdalny (16.01.2021): \url{https://pl.reactjs.org/docs/reconciliation.html}"
}

@manual{ref_cra_doc,
    title = "Create React App",
    note  = "Dostęp zdalny (16.01.2021): \url{https://github.com/facebook/create-react-app}"
}
@manual{ref_eslint_doc,
    title = "Eslint",
    note  = "Dostęp zdalny (16.01.2021): \url{https://eslint.org/}"
}
@manual{ref_ts_doc,
    title = "Typescript",
    note  = "Dostęp zdalny (16.01.2021): \url{https://www.typescriptlang.org/}"
}
@manual{ref_graphql_doc,
    title = "Graphql",
    note  = "Dostęp zdalny (16.01.2021): \url{https://graphql.org/}"
}
@manual{ref_gqlKotlin_doc,
    title = "Graphql-Kotlin",
    note  = "Dostęp zdalny (16.01.2021): \url{https://expediagroup.github.io/graphql-kotlin/docs/getting-started.html}"
}
@manual{ref_postgre_doc,
    title = "PostgreSQL",
    note  = "Dostęp zdalny (16.01.2021): \url{https://www.postgresql.org/}"
}
@manual{ref_spring_doc,
    title = "Spring",
    note  = "Dostęp zdalny (16.01.2021): \url{https://spring.io/}"
}
@manual{ref_kotlin_doc,
    title = "Kotlin",
    note  = "Dostęp zdalny (16.01.2021): \url{https://kotlinlang.org/}"
}

@unpublished{ref_hex_doc,
    author = "Alistair Cockburn",
    title  = "Hexagonal architecture",
    note   = "Dostęp zdalny (16.01.2021): \url{https://alistair.cockburn.us/hexagonal-architecture/}",
    year   = "2005"
}

% EXAMPLES
% Artykuł w recenzowanym czasopiśmie.
@article{szczypiorski2015,
    author    = "Szczypiorski, K. and Janicki, A. and Wendzel, S",
    title     = "{T}he {G}ood, {T}he {B}ad {A}nd {T}he {U}gly: {E}valuation of {W}i-{F}i {S}teganography",
    journal   = "Journal of Communications",
    volume    = "10",
    number    = "10",
    pages     = "747--752",
    publisher = "Journal of Communications (JCM)",
    year      = "2015",
}

% Książka.
@book{goossens93,
    author    = "Michel Goossens and Frank Mittelbach and Alexander Samarin",
    title     = "The LaTeX Companion",
    publisher = "Addison-Wesley",
    address   = "Reading, Massachusetts",
    year      = "1993",
}

% Fragment książki (np. zakres stron).
@inbook{wang97,
    author      = "Hao ang",
    title       = "A Logical Journey: From G{\"o}del to Philosophy.",
    publisher   = "A Bradford Book",
    pages       = "316",
    year        = "1997"
}

% Fragment książki (np. esej), posiadający własny tytuł.
@incollection{goedel95,
    author      = "Kurt G{\"o}del",
    title       = "Texts relating to the ontological proof",
    booktitle   = "Unpublished Essays and Lectures",
    publisher   = "Oxford University Press",
    pages       = "429--437",
    year        = "1995",
}

% Publikacja konferencyjna.
@inproceedings{benzmuller2014,
    author       = "{Ch}. Benzmuller and B. W. Paleo",
    title        = "Automating {G\"o}del’s {O}ntological {P}roof of {G}od’s {E}xistence with {H}igher-order {A}utomated {T}heorem {P}rovers",
    booktitle    = "European	Conference on Artificial Intelligence",
    publisher    = "IOS Press",
    howpublished = "Dostęp zdalny (10.04.2019): \url{http://page.mi.fu-berlin.de/cbenzmueller/papers/C40.pdf}",
    year         = "2014",
}

% Raport techniczny.
@techreport{duqu2011,
    author      = "Bencsáth, B. and Pék, G. and Buttyán, L. and Félegyházi M.",
    title       = "{D}uqu: {A} {S}tuxnet-like malware found in the wild",
    institution = "Laboratory of Cryptography and System Security, Hungary",
    year        = "2011"
}

% Specyfikacja techniczna.
@manual{shs2015,
    title        = "{FIPS} 180-4: {S}ecure {H}ash {S}tandard ({SHS})",
    howpublished = "Dostęp zdalny (13.03.2019): \url{https://nvlpubs.nist.gov/nistpubs/FIPS/NIST.FIPS.180-4.pdf}",
    year         = "2015"
}

% Praca magisterska.
@mastersthesis{wozniak2018,
    author = "Woźniak, Piotr",
    title  = "{P}rogramowanie kwadratowe w usuwaniu efektu rozmycia ruchu w fotografii cyfrowej",
    school  = "Wydział Elektroniki i Technik Informacyjnych, Politechnika Warszawska",
    year   = "2018",
}

% Nieopublikowany artykuł, dostępny np. tylko w internecie.
@unpublished{koons2005,
    author = "Koons, Robert C.",
    title  = "{S}obel on {G\"o}del’s {O}ntological {P}roof",
    note   = "Dostęp zdalny (25.04.2019): \url{http://www.robkoons.net/media/69b0dd04a9d2fc6dffff80b4ffffd524.pdf}",
    year   = "2005"
}
