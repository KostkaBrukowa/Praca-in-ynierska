\newpage % Rozdziały zaczynamy od nowej strony.
\section{Wprowadzenie}

Zdarza się w naszym życiu, że razem z innymi osobami kupujemy rzeczy, za które płaci tylko jedna osoba. W takiej sytuacji dosyć uciążliwe może być proszenie innych osób o zwrot kwoty, obliczanie owej kwoty, którą należy zwrócić płatnikowi, oraz śledzenie wszystkich płatności i wydatków, w których uczestniczyliśmy. W takiej sytuacji dobrym pomysłem byłoby użycie aplikacji lub programu komputerowego, który pomagałby nam w śledzeniu takich elementów naszego życia. W ramach tej pracy została utworzona właśnie taka aplikacja i została nazwana Aprint.

\subsection{Opis przypadku biznesowego}
\subsubsection{Opis hipotetycznej sytuacji}

Aby bardziej zilustrować przypadek biznesowy, możemy wyobrazić sobie sytuację, w której dwóch współlokatorów (nazwijmy ich Rafał i Kacper) mieszka w jednym mieszkaniu i obydwaj wydają pieniądze na rzeczy potrzebne na utrzymanie całego mieszkania. Pewnego dnia Rafał kupuje środki czystości. Jak możemy się domyślić, środki czystości będą używane przez Kacpra i Rafała, natomiast w tej hipotetycznej sytuacji tylko Rafał zapłacił w całości za owe produkty. Z tego powodu Kacper jest winny części kwoty Rafałowi. Obydwaj używając aplikacji, mogą w prosty sposób rozliczyć się z tego wydatku.

\subsubsection{Użycie aplikacji}

Na samym początku Rafał musi utworzyć obiekt wydatku na stronie internetowej, używając do tego dedykowanego formularza, wypełniając takie informacje jak nazwa wydatku, jego opis, datę kiedy ten wydatek został stworzony oraz osoby, które razem z nim z brali udział w wydatku (w tym wydatku tylko Kacpra).

Gdy wydatek zostanie utworzony, Kacper będzie mógł potwierdzić lub odrzucić udział w owej transakcji. Będzie mógł to zrobić na dedykowanym ekranie płatności, na którym będą odpowiednie przyciski oraz informacje dotyczące płatności, takie jak opis, data utworzenia wydatku itp. Jeżeli Kacper potwierdzi to, że brał udział w tym wydatku, będzie mógł nacisnąć przycisk \emph{potwierdzam płatność}, a jeżeli się z tym nie zgadza, naciśnie przycisk \emph{odrzucam płatność}.

Gdy wszyscy uczestnicy wydatku (w tym przypadku tylko Kacper) potwierdzi lub odrzuci udział, inicjator wydatku będzie mógł zatwierdzić uczestników, czyli kliknąć przycisk na ekranie wydatku o treści "potwierdzam użytkowników". To potwierdzenie nie mogło zostać zaimplementowane automatycznie, ponieważ inicjator wydatku może nie zgodzić się z uczestnikami i w takim wydatku może z nimi porozmawiać lub wyjaśnić niespójności.

Następnym krokiem tego scenariusza jest już sama akcja zapłaty za wydatek. Uczestnik wydatku Kacper może przejść na ekran płatności, na którym będzie już obliczona kwota, którą Kacper musi zapłacić Rafałowi. Po płatności ze strony Kacpra będzie mógł on kliknąć przycisk "potwierdzam płatność" i uregulować w ten sposób należność.

Ostatnim elementem scenariusza jest zakończenie wydatku przez inicjatora wtedy, kiedy wszyscy uczestnicy uregulują swoje płatności. Będzie on mógł to zrobić na ekranie wydatku po kliknięciu w przycisk zakończ wydatek.

\subsection{Układ pracy}
Praca jest podzielona na 4 rozdziały, gdzie w pierwszym zostaną opisane wymagania, jakie aplikacja do zarządzania wspólnymi wydatkami powinna spełniać. W drugim rozdziale zostaną opisane użyte technologie i podejścia, które zostały użyte podczas pisania aplikacji. W kolejnym rozdziale zostaną opisane różne sposoby automatycznego testowania aplikacji internetowej. W ostatnim rozdziale zostanie przedstawiony wygląd i poszczególne ekrany aplikacji.


